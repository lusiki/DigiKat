% Options for packages loaded elsewhere
\PassOptionsToPackage{unicode}{hyperref}
\PassOptionsToPackage{hyphens}{url}
%
\documentclass[
]{article}
\usepackage{amsmath,amssymb}
\usepackage{iftex}
\ifPDFTeX
  \usepackage[T1]{fontenc}
  \usepackage[utf8]{inputenc}
  \usepackage{textcomp} % provide euro and other symbols
\else % if luatex or xetex
  \usepackage{unicode-math} % this also loads fontspec
  \defaultfontfeatures{Scale=MatchLowercase}
  \defaultfontfeatures[\rmfamily]{Ligatures=TeX,Scale=1}
\fi
\usepackage{lmodern}
\ifPDFTeX\else
  % xetex/luatex font selection
\fi
% Use upquote if available, for straight quotes in verbatim environments
\IfFileExists{upquote.sty}{\usepackage{upquote}}{}
\IfFileExists{microtype.sty}{% use microtype if available
  \usepackage[]{microtype}
  \UseMicrotypeSet[protrusion]{basicmath} % disable protrusion for tt fonts
}{}
\makeatletter
\@ifundefined{KOMAClassName}{% if non-KOMA class
  \IfFileExists{parskip.sty}{%
    \usepackage{parskip}
  }{% else
    \setlength{\parindent}{0pt}
    \setlength{\parskip}{6pt plus 2pt minus 1pt}}
}{% if KOMA class
  \KOMAoptions{parskip=half}}
\makeatother
\usepackage{xcolor}
\usepackage[margin=1in]{geometry}
\usepackage{graphicx}
\makeatletter
\def\maxwidth{\ifdim\Gin@nat@width>\linewidth\linewidth\else\Gin@nat@width\fi}
\def\maxheight{\ifdim\Gin@nat@height>\textheight\textheight\else\Gin@nat@height\fi}
\makeatother
% Scale images if necessary, so that they will not overflow the page
% margins by default, and it is still possible to overwrite the defaults
% using explicit options in \includegraphics[width, height, ...]{}
\setkeys{Gin}{width=\maxwidth,height=\maxheight,keepaspectratio}
% Set default figure placement to htbp
\makeatletter
\def\fps@figure{htbp}
\makeatother
\setlength{\emergencystretch}{3em} % prevent overfull lines
\providecommand{\tightlist}{%
  \setlength{\itemsep}{0pt}\setlength{\parskip}{0pt}}
\setcounter{secnumdepth}{5}
\ifLuaTeX
  \usepackage{selnolig}  % disable illegal ligatures
\fi
\usepackage{bookmark}
\IfFileExists{xurl.sty}{\usepackage{xurl}}{} % add URL line breaks if available
\urlstyle{same}
\hypersetup{
  pdftitle={Baza podataka - Katolički digitalni medijski prostor u Hrvatskoj (2021-2024)},
  hidelinks,
  pdfcreator={LaTeX via pandoc}}

\title{Baza podataka - Katolički digitalni medijski prostor u Hrvatskoj
(2021-2024)}
\author{}
\date{\vspace{-2.5em}2025-07-23}

\begin{document}
\maketitle

{
\setcounter{tocdepth}{2}
\tableofcontents
}
\section{Pregled baze podataka}\label{pregled-baze-podataka}

Baza podataka sadrži \textbf{258,757 zapisa} medijskih objava
prikupljenih tijekom 2021/22/23. godine, fokusiranih na katoličke teme i
sadržaje u hrvatskim medijima. Baza predstavlja sveobuhvatan korpus za
analizu medijskog diskursa, sentimenta i angažmana publike u domeni
religijskih tema.

\begin{longtable}[t]{ll}
\caption{\label{tab:load-data}Osnovne karakteristike dataseta}\\
\toprule
Karakteristika & Vrijednost\\
\midrule
Broj zapisa & 258,757\\
Broj varijabli & 49\\
Vremenski period & 2021 - 2024 godina\\
Format & R data.table\\
Glavni jezik & Hrvatski (hr)\\
\addlinespace
Geografski opseg & Hrvatska (HR)\\
\bottomrule
\end{longtable}

\subsection{Struktura podataka}\label{struktura-podataka}

\begin{verbatim}
## Classes 'data.table' and 'data.frame': 258757 obs. of 49 variables:
\end{verbatim}

\begin{verbatim}
## $ DATE                 : chr [1:258757] '2021-01-02' '2021-01-02' ...
\end{verbatim}

\begin{verbatim}
## $ TIME                 : chr [1:258757] '23:36:00' '23:28:34' ...
\end{verbatim}

\begin{verbatim}
## $ TITLE                : chr [1:258757] 'Župa Gospe Brze Pomoći...' ...
\end{verbatim}

\begin{verbatim}
## $ AUTO_SENTIMENT       : chr [1:258757] 'positive' 'neutral' 'negative' ...
\end{verbatim}

\begin{verbatim}
## $ REACH                : num [1:258757] 21 48 3300 312 ...
\end{verbatim}

\begin{verbatim}
## ... [dodatnih 44 varijabli]
\end{verbatim}

\section{Opis varijabli}\label{opis-varijabli}

\subsection{Vremenske varijable}\label{vremenske-varijable}

\begin{longtable}[t]{lll}
\caption{\label{tab:temporal-vars}Vremenske varijable}\\
\toprule
Varijabla & Tip & Opis\\
\midrule
DATE & character & Datum objave u formatu YYYY-MM-DD\\
TIME & character & Vrijeme objave u formatu HH:MM:SS\\
year & numeric & Godina izvlučena iz datuma\\
\bottomrule
\end{longtable}

\subsection{Sadržaj i metapodaci}\label{sadrux17eaj-i-metapodaci}

\begin{longtable}[t]{lll}
\caption{\label{tab:content-vars}Sadržaj i metapodaci}\\
\toprule
Varijabla & Tip & Opis\\
\midrule
TITLE & character & Naslov članka/objave\\
FULL\_TEXT & character & Potpuni tekst objave(dostupan isključivo na zahtjev)\\
MENTION\_SNIPPET & character & Isječak teksta koji sadrži ključne riječi\\
AUTHOR & character & Autor objave (ako je dostupan)\\
FROM & character & Izvor/domena web stranice\\
\addlinespace
URL & character & Potpuna URL adresa objave\\
URL\_PHOTO & character & URL fotografije povezane s objavom\\
\bottomrule
\end{longtable}

\subsection{Kategorizacija i
označavanje}\label{kategorizacija-i-oznaux10davanje}

\begin{longtable}[t]{lll}
\caption{\label{tab:categorization-vars}Kategorizacija i označavanje}\\
\toprule
Varijabla & Tip & Opis\\
\midrule
SOURCE\_TYPE & factor & Tip izvora (web, youtube, facebook, twitter itd.)\\
GROUP\_NAME & character & Naziv grupe za praćenje\\
KEYWORD\_NAME & character & Naziv ključne riječi\\
FOUND\_KEYWORDS & character & Pronađene ključne riječi u tekstu\\
TAGS & logical & Dodatne oznake (trenutno prazno)\\
\addlinespace
LANGUAGES & character & Jezik objave (hr, bs)\\
LOCATIONS & character & Geografska lokacija (HR)\\
\bottomrule
\end{longtable}

\subsection{Sentiment analiza}\label{sentiment-analiza}

\begin{longtable}[t]{lll}
\caption{\label{tab:sentiment-vars}Sentiment analiza}\\
\toprule
Varijabla & Tip & Opis\\
\midrule
AUTO\_SENTIMENT & character & Automatski detektirani sentiment (positive/neutral/negative)\\
MANUAL\_SENTIMENT & logical & Ručno označen sentiment (trenutno prazno)\\
\bottomrule
\end{longtable}

\subsection{Metrike angažmana}\label{metrike-angaux17emana}

\begin{longtable}[t]{lll}
\caption{\label{tab:engagement-vars}Metrike angažmana}\\
\toprule
Varijabla & Tip & Opis\\
\midrule
REACH & numeric & Doseg objave (broj ljudi koji je vidjelo)\\
VIRALITY & numeric & Indeks viralnosti\\
ENGAGEMENT\_RATE & numeric & Stopa angažmana (\%)\\
INTERACTIONS & numeric & Ukupan broj interakcija\\
FOLLOWERS\_COUNT & numeric & Broj pratitelja autora\\
\bottomrule
\end{longtable}

\subsection{Specifične reakcije
(Facebook)}\label{specifiux10dne-reakcije-facebook}

\begin{longtable}[t]{lll}
\caption{\label{tab:facebook-vars}Specifične reakcije (Facebook)}\\
\toprule
Varijabla & Tip & Opis\\
\midrule
LIKE\_COUNT & numeric & Broj 'like' reakcija\\
LOVE\_COUNT & numeric & Broj 'love' reakcija\\
WOW\_COUNT & numeric & Broj 'wow' reakcija\\
HAHA\_COUNT & numeric & Broj 'haha' reakcija\\
SAD\_COUNT & numeric & Broj 'sad' reakcija\\
\addlinespace
ANGRY\_COUNT & numeric & Broj 'angry' reakcija\\
COMMENT\_COUNT & numeric & Broj komentara\\
SHARE\_COUNT & numeric & Broj dijeljenja\\
TOTAL\_REACTIONS\_COUNT & numeric & Ukupan broj svih reakcija\\
\bottomrule
\end{longtable}

\section{Kvaliteta i kompletnost
podataka}\label{kvaliteta-i-kompletnost-podataka}

\subsection{Pregled nedostajućih
vrijednosti}\label{pregled-nedostajuux107ih-vrijednosti}

\begin{longtable}[t]{lrl}
\caption{\label{tab:missing-data}Pregled nedostajućih vrijednosti u ključnim varijablama}\\
\toprule
Varijabla & Nedostaje (\%) & Razlog\\
\midrule
AUTHOR & 35.2 & Nije uvijek dostupno od izvora\\
TAGS & 100.0 & Funkcionalnost nije implementirana\\
MANUAL\_SENTIMENT & 100.0 & Ručno označavanje nije provedeno\\
FOLLOWERS\_COUNT & 68.5 & Ovisi o platformi i dostupnosti\\
REDDIT\_SCORE & 92.1 & Specifično za Reddit objave\\
\addlinespace
VIEW\_COUNT & 85.3 & Specifično za video sadržaj\\
\bottomrule
\end{longtable}

\subsection{Statistički sažetak numeričkih
varijabli}\label{statistiux10dki-saux17eetak-numeriux10dkih-varijabli}

\begin{longtable}[t]{lrrrrr}
\caption{\label{tab:numeric-summary}Statistički sažetak ključnih numeričkih varijabli}\\
\toprule
Varijabla & Mean & Median & SD & Min & Max\\
\midrule
REACH & 2543.2 & 312.0 & 8734.5 & 0 & 125000.0\\
INTERACTIONS & 67.8 & 8.0 & 245.6 & 0 & 5847.0\\
ENGAGEMENT\_RATE & 4.2 & 1.8 & 8.7 & 0 & 89.5\\
LIKE\_COUNT & 52.1 & 6.0 & 187.3 & 0 & 3245.0\\
INFLUENCE\_SCORE & 3.2 & 3.0 & 2.1 & 1 & 10.0\\
\bottomrule
\end{longtable}

\section{Primjeri korištenja}\label{primjeri-koriux161tenja}

\subsection{Osnovne analize}\label{osnovne-analize}

\begin{longtable}[t]{l>{\raggedright\arraybackslash}p{35%}l>{\raggedright\arraybackslash}p{25%}}
\caption{\label{tab:basic-analysis-table}Osnovne analize - pregled kodova i rezultata}\\
\toprule
Tip analize & R kod & Rezultat & Sortiranje\\
\midrule
Analiza sentimenta po izvorima & sentiment\_by\_source <- dta[, .N, by = .(FROM, AUTO\_SENTIMENT)] & Broj objava po izvoru i sentimentu & sentiment\_by\_source[order(-N)]\\
Trendovi kroz vrijeme & temporal\_trends <- dta[, .N, by = .(year, month = substr(DATE, 6, 7))] & Broj objava po mjesecima & temporal\_trends[order(year, month)]\\
Top izvori po angažmanu & top\_sources <- dta[, .(avg\_engagement = mean(ENGAGEMENT\_RATE, na.rm = TRUE)), by = FROM] & Prosječni angažman po izvoru & top\_sources[order(-avg\_engagement)]\\
\bottomrule
\end{longtable}

\subsection{Napredne analize}\label{napredne-analize}

\begin{longtable}[t]{ll>{\raggedright\arraybackslash}p{30%}l}
\caption{\label{tab:advanced-analysis-table}Napredne analize - detaljni pregled metoda}\\
\toprule
Analiza & Potrebne biblioteke & Ključni kod & Očekivani output\\
\midrule
Tokenizacija teksta & tidytext, dplyr & unnest\_tokens(word, TITLE) & Pojedinačne riječi iz naslova\\
Čišćenje stop riječi & tidytext & anti\_join(stop\_words) & Filtrirane značajne riječi\\
Brojanje riječi po sentimentu & dplyr, tidytext & count(word, AUTO\_SENTIMENT, sort = TRUE) & Frekvencija riječi po sentimentu\\
Sentiment scoring & dplyr, case\_when & summarise(sentiment\_score = sum(n * case\_when(...))) & Numerički sentiment score\\
Wordcloud generiranje & wordcloud, RColorBrewer & wordcloud(words, freq, colors = brewer.pal(8, 'Dark2')) & Vizualna reprezentacija\\
\bottomrule
\end{longtable}

\section{Tehnički detalji}\label{tehniux10dki-detalji}

\subsection{Izvorni format podataka}\label{izvorni-format-podataka}

\begin{longtable}[t]{l>{\raggedright\arraybackslash}p{40%}>{\raggedright\arraybackslash}p{35%}}
\caption{\label{tab:technical-details-table}Tehnički detalji dataseta i preporučeni alati}\\
\toprule
Kategorija & Vrijednost/Opis & Napomene\\
\midrule
\cellcolor[HTML]{f8f9fa}{Izvorne datoteke} & \cellcolor[HTML]{f8f9fa}{op\_e\_[datum-raspon].xlsx} & \cellcolor[HTML]{f8f9fa}{Batch obrada po vremenskim periodima}\\
\cellcolor[HTML]{f8f9fa}{Format obrade} & \cellcolor[HTML]{f8f9fa}{R data.table} & \cellcolor[HTML]{f8f9fa}{Optimizirano za velike podatke}\\
\cellcolor[HTML]{f8f9fa}{Kodiranje} & \cellcolor[HTML]{f8f9fa}{UTF-8} & \cellcolor[HTML]{f8f9fa}{Podrška za hrvatske znakove}\\
\cellcolor[HTML]{f8f9fa}{Separatori} & \cellcolor[HTML]{f8f9fa}{Automatski detektirani} & \cellcolor[HTML]{f8f9fa}{Excel format automatski parsiran}\\
\cellcolor[HTML]{f8f9fa}{Nedostajuće vrijednosti} & \cellcolor[HTML]{f8f9fa}{NA} & \cellcolor[HTML]{f8f9fa}{Standardno R označavanje}\\
\addlinespace
\cellcolor[HTML]{e3f2fd}{Preporučena biblioteka - Manipulacija} & \cellcolor[HTML]{e3f2fd}{data.table - za brzu manipulaciju velikih dataset-a} & \cellcolor[HTML]{e3f2fd}{Brzina: 10-100x brža od base R}\\
\cellcolor[HTML]{e3f2fd}{Preporučena biblioteka - Sintaksa} & \cellcolor[HTML]{e3f2fd}{dplyr - za čišću i čitljiviju sintaksu} & \cellcolor[HTML]{e3f2fd}{Kompatibilnost s tidyverse ekosystemom}\\
\cellcolor[HTML]{e3f2fd}{Preporučena biblioteka - Vizualizacija} & \cellcolor[HTML]{e3f2fd}{ggplot2 - za profesionalne vizualizacije} & \cellcolor[HTML]{e3f2fd}{Grammar of graphics pristup}\\
\cellcolor[HTML]{e3f2fd}{Preporučena biblioteka - Datumi} & \cellcolor[HTML]{e3f2fd}{lubridate - za rad s datumskim formatima} & \cellcolor[HTML]{e3f2fd}{Timezone aware operacije}\\
\cellcolor[HTML]{e3f2fd}{Preporučena biblioteka - Tekst} & \cellcolor[HTML]{e3f2fd}{stringr - za manipulaciju i analizu teksta} & \cellcolor[HTML]{e3f2fd}{Regex podrška za složene operacije}\\
\bottomrule
\end{longtable}

\subsection{Napomene o performansama}\label{napomene-o-performansama}

\begin{longtable}[t]{lll}
\caption{\label{tab:performance-notes}Preporuke za optimalne performanse}\\
\toprule
Operacija & Preporučeni pristup & Očekivano vrijeme\\
\midrule
Čitanje podataka & fread() za brže učitavanje & < 30 sekundi\\
Grupiranje i agregacija & data.table sintaksa [, .N, by=] & < 5 sekundi\\
Filtriranje velikih tekstova & Koristiti grep() s fixed=TRUE & 10-60 sekundi\\
Sortiranje po datumu & Konvertirati DATE u Date klasu & < 10 sekundi\\
Analiza sentimenta & Koristiti existirajuće AUTO\_SENTIMENT & 1-5 minuta\\
\bottomrule
\end{longtable}

\subsection{Preuzmi bazu podataka}\label{preuzmi-bazu-podataka}

📊 \textbf{Kaggle Dataset}:
\href{https://www.kaggle.com/datasets/lukasikic/croatian-catholic-digital-media-space/data1}{Croatian
Catholic Media Space 2021- 2024}

\section{Licence i citiranje}\label{licence-i-citiranje}

Molimo citirajte ovu bazu u svojim radovima koristeći sljedeći format:

\begin{quote}
\textbf{{[}Šikić, Luka/Hrvatsko katoličko sveučilište{]}}. (2025).
\emph{Katolički digitalni medijski prostor u Hrvatskoj 2025}. Dataset
sadrži 258,757 medijskih objava iz hrvatskih medija. Pristupljeno:
2025-07-23.
\end{quote}

\subsection{Dodatni resursi}\label{dodatni-resursi}

\begin{itemize}
\tightlist
\item
  \textbf{GitHub repozitorij}:
  {[}\url{https://github.com/lusiki/DigiKat}{]}
\item
  \textbf{Dokumentacija}:
  {[}\url{https://lusiki.github.io/DigiKat/baza.html}{]}
\item
  \textbf{Kontakt}:
  {[}\href{mailto:luka.sikic@unicath.hr}{\nolinkurl{luka.sikic@unicath.hr}}{]}
\item
  \textbf{ORCID}: {[}0009-0006-3519-0272{]}
\end{itemize}

\begin{center}\rule{0.5\linewidth}{0.5pt}\end{center}

\textbf{Zadnja ažurirana}: 2025-07-23\\
\textbf{Verzija}: 1.0\\
\textbf{R verzija}: R version 4.2.2 (2022-10-31 ucrt)

\end{document}
